\subsection{F�r das Spiel registrieren}
\begin{description}
\item [Beschreibung] Der Spieler w�hlt einen Benutzernamen und ein Passwort. Ist der Benutzername noch nicht vergeben m�ssen weitere Angaben gemacht werden, um die Registrierung abzuschlie�en (Emailadresse, gew�hltes Volk). Bei Abschluss der Registrierung erh�lt der Spieler eine Email mit einer Aktivierungsanweisung.
\item [Beispiel]
\item [Bemerkungen]
\item [Implementationsdetails]Der erste Teil der Reistrierung enth�lt einen Dialog mit zwei Textfelden, in denen der Spieler einen Benutzernamen und ein Passwort w�hlt. Nach dem Versenden der Informationen wird in der Datenbank �berpr�ft, ob der Benutzername bereits vergeben ist, falls ja, muss der Spieler nachbessern. Kann der Name gew�hlt werden, m�ssen weitere relevante Informationen vergeben werden, die f�r das Spiel wichtig sind. Auch hierzu dient wieder ein Formular mit verschiedenen Text- und Auswahlfeldern. Die Informationen werden versammelt verschickt und in die Datenbank eingetragen. Der Spieler ist somit registriert und erh�lt eine Email mit einem Aktivierungslink.
\item [Aufgaben]
  \begin{itemize}
  \item Registrierungsseiten mit Eingabefeldern bauen
  \item Datenbankabfrage, ob Benutzer bereits vergeben
  \item Passwortkontrolle, ob beide Passw�rter gleich
  \item �bermittelte Daten in der Datenbank ablegen
  \item Aktivierungslink bauen und dem Spieler per Email zukommen lassen
  \item Best�tigungsseite anzeigen
  \end{itemize}
\end{description}
