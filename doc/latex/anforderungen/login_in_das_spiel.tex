\subsection{Login in das Spiel}
\begin{description}
\item [Beschreibung] Der Anwender l�dt die Seite im Browser und kommt zu einer Eingabeaufforderung. Dort muss er sich mittels Benutzername und Passwort authentifizieren. Ein erfolgreiches Login f�hrt ihn in das Spiel, ein Misserfolg zu einer Fehlerseite.
\item [Beispiel]
\item [Bemerkungen]
\item [Implementationsdetails] Das Loginfenster besteht aus zwei Texteingabefeldern, eines f�r den Benutzernamen, eines f�r das Passwort und einem Button, der die Anfrage absendet. Anschlie�end wird �berpr�ft, ob der Benutzer und das Passwort in der Datenbank gefunden werden k�nnen. Sind diese vorhanden, wird ein Sessioncookie mit Benutzerid und Benutzername gesetzt, damit keine erneute Abfrage n�tig wird. Zus�tzlich werden die aktuelle Zeit und die aktuelle IP des Benutzers in der Datenbank abgelegt. Grunds�tzlich muss noch gepr�ft werden, ob der Client des Anwenders Cookies unterst�tzt.
\item [Aufgaben]
  \begin{itemize}
  \item Datenbankabfrage Benutzer-/Passwortabgleich
  \item Cookiepr�f und -speicherroutine
  \item Datenbankupdatefunktion f�r Datum und IP
  \end{itemize}
\end{description}
